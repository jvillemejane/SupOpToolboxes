%----------------------------------------------------------------------------------------
%	SHORT INTRODUCTION
%----------------------------------------------------------------------------------------
Les \textbf{expériences scientifiques}, les \textbf{essais industriels} sur des systèmes ou bien encore des \textbf{résultats de simulation} produisent énormément de \textbf{données}. 
Ces données sont souvent sauvegardées sous forme de \textbf{fichiers formattés} (format normalisé ou interne aux entreprises/laboratoires).

Il est alors indispensable de pouvoir \textbf{afficher les données} contenues dans ce type de fichier de manière claire et sans ambiguïté, avant d'en \textbf{extraire des informations pertinentes} par un traitement adapté.

\medskip

Vous traiterez dans cette séquence une \textbf{information modulée en amplitude}, acquise par un \textbf{oscilloscope numérique} et stockée dans un \textbf{fichier de type tableur}.